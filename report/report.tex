
\documentclass[10pt, titlepage, oneside, a4paper]{article}
\usepackage[T1]{fontenc}
\usepackage{wrapfig}
\usepackage[english]{babel}
\usepackage{amssymb, graphicx, fancyhdr}
\usepackage{url}
\usepackage{listings}
\usepackage{verbatim}
\usepackage{verbatimbox}
\usepackage{varioref}
\addtolength{\textheight}{20mm}
\addtolength{\voffset}{-5mm}
\renewcommand{\sectionmark}[1]{\markleft{#1}}

\newcommand{\Section}[1]{\section{#1}\vspace{-8pt}}
\newcommand{\Subsection}[1]{\vspace{-4pt}\subsection{#1}\vspace{-8pt}}
\newcommand{\Subsubsection}[1]{\vspace{-4pt}\subsubsection{#1}\vspace{-8pt}}
	

\newcounter{appendixpage}

\newenvironment{appendices}{
	\setcounter{appendixpage}{\arabic{page}}
	\stepcounter{appendixpage}
}{
}

\newcommand{\appitem}[2]{
	\stepcounter{section}
	\addtocontents{toc}{\protect\contentsline{section}{\numberline{\Alph{section}}#1}{\arabic{appendixpage}}}
	\addtocounter{appendixpage}{#2}
}

\newcommand{\appsubitem}[2]{
	\stepcounter{subsection}
	\addtocontents{toc}{\protect\contentsline{subsection}{\numberline{\Alph{section}.\arabic{subsection}}#1}{\arabic{appendixpage}}}
	\addtocounter{appendixpage}{#2}
}


\def\inst{Computing Science}
\def\typeofdoc{Obligatory Exercise}
\def\course{Programing Languages}
\def\pretitle{Short Message Service}
\def\title{Obligatory Exercise 2}
\def\name{Simon Nilsson}
\def\username{c09snn}
\def\email{\username{}@cs.umu.se}
\def\path{edu/5dv086/lab2}
\def\graderA{Jan Erik Mostr�m}
\def\graderB{Petter Ericson}
\def\graderC{}

\def\fullpath{\raisebox{1pt}{$\scriptstyle \sim$}\username/\path}

\begin{document}


	\begin{titlepage}
		\thispagestyle{empty}
		\begin{large}
			\begin{tabular}{@{}p{0.93\textwidth}@{}}
				\textbf{UME� UNIVERSITY \hfill \today} \\
				\textbf{Department of \inst} \\
				\textbf{\typeofdoc} \\
			\end{tabular}
		\end{large}
		\vspace{10mm}
		\begin{center}
			\Large{\textbf{\course}}\\
			\vspace{10mm}
			\huge{\textbf{\title}} \\
			\LARGE{\pretitle} \\

			\vspace{15mm}
			\begin{large}
				\begin{tabular}{ll}
					\textbf{Name} & \name \\
					\textbf{E-mail} & \texttt{\email} \\
					\textbf{Path} & \texttt{\fullpath} \\
				\end{tabular}
			\end{large} \\
			\vspace{20mm}

			%\LARGE{\textbf{Resubmission 1}} \\
			\vspace{10mm}
			%\Large{\textbf{Change log}}\\
			
		%	\begin{large}
      
		%		\begin{itemize}
		%		 \item add\_vertex now returns vertex id \\
		%		\item The Java implementation of add\_edge can \\
		%		no longer cause the DAG to have cycles
		%		\end{itemize}
		%	 \end{large}
			 
			
			\vfill
			\large{\textbf{Grader}}\\
			\mbox{\large{\graderC}}\\
			\mbox{\large{\graderA}}\\
			\mbox{\large{\graderB}}
		\end{center}
	\end{titlepage}


	% fixar sidfot
	\lfoot{\footnotesize{\name, \email}}
	\rfoot{\footnotesize{\today}}
	\lhead{\sc\footnotesize\title}
	\rhead{\nouppercase{\sc\footnotesize\leftmark}}
	\pagestyle{fancy}
	\renewcommand{\headrulewidth}{0.2pt}
	\renewcommand{\footrulewidth}{0.2pt}


	\pagenumbering{roman}
%	\tableofcontents
	

	\newpage

	\pagenumbering{arabic}


	\setlength{\parindent}{0pt}

	\setlength{\parskip}{10pt}

\Section{Introduction}
T9 is a technology used in phones to quickly type text without using a keyboard, even though most 
new phones uses a virtual keyboard via touch screens. The process is based on having many letters and symbols 
on each key. Most of the latter systems, such as T9, were dictionary based and reduces the number 
of key buttons pressed by mapping the input to the dictionary. The dictionary also picks, unless 
the user choose something else, the most common word. 

The goal is to implement a T9 system using Haskell. The system will take a message 
input and return a minimum sequence of key pressed that produces a given message for a particular
dictionary augmented with word counts. 

The system must have the following properties:

\begin{itemize}
	\item{A message is a string of upper-case letters A-Z and blanks.}
	\item{An input string is a sequence of digits (0,2-9) and the special character \textasciicircum.}
	\item{A prefix is a sequence of digits (2-9) representing letters.}
	%\item{The character \textasciicircum appended to a prefix modifies that prefix.}
	\item{A prefix can be modified any number of times.}
	\item{Each (modified) prefix is mapped into a word as follows.}
	\begin{itemize}
		\item{The prefix together with a dictionary determines a set of possible words.}
		\item{The possible words are sorted in decreasing order of frequency.}
		\item{Ties are broken by ordering the words lexicographically.}
		\item{If there are no possible words, the prefix maps into the special character ?.}
		\item{A prefix which has been modified N times maps into the (N+1)-th word in the list.}
		\item{If (N+1) exceeds the length of the list, then it wraps around the list.}
	\end{itemize}
\end{itemize}

This implementation uses a pre-defined dictionary that can be found here:
\url{http://www8.cs.umu.se/kurser/5DV086/VT14/resources/Dictionary.hs}

The original specification can be found at:\\
\url{http://www8.cs.umu.se/kurser/5DV086/VT14/xlab2.html}

\Section{User Guide}
The implementation can be found at \texttt{c09snn/edu/5DV086/lab2} and also on \texttt{GitHub}\footnote{Type of revision 
control, also known as version control and source control. It is the management of changes to documents, computer programs, 
large web sites, and other collections of information.} at the address
\url{https://github.com/SNilsson/5dv086-lab2} The solution 
requires a file named \texttt{Dictionary.hs} to work correctly. For verification and testing purpose 
the files \texttt{Dictionary.hs}, \texttt{test.hs} and \texttt{Messages.hs} is added to the final submission. 

The implementation follows the specification provided with additional modifications: 

The implementation sorts the sequences of words by their priority and then by name.

To run the T9 implementation the user must have \texttt{ghci}\footnote{GHC is a state-of-the-art, open source, compiler 
and interactive environment for the functional language Haskell.
GHCi is GHC's interactive environment, in which Haskell expressions can be interactively evaluated and programs can be interpreted.} 
installed and through it load \texttt{T9.hs}.
Then execute the method \texttt{get\_t9\_sequence} that accepts a string as input. When imported from other files
the only accessible method that can be run is \texttt{get\_t9\_sequence} and it should be the only one the user
executes.

\texttt{test.hs} can be used to check all the messages in \texttt{Messages.hs}.

\Section{Implementation}
The implementation is fairly straight forward with few algorithms. \texttt{get\_t9} uses other hidden 
functions that helps it decipher:
\begin{itemize}
\item{The amount of keys pressed to generate the word}
\item{The shortest sequence of keys pressed to generate the word}
\item{How many \textasciicircum that should be used to get the correct word or word sequence given}
\end{itemize}

\subsection{ \texttt{get\_shortest\_sequence} }
This function takes a sequence of characters, i.e a word and a dictionary as input and returns a sequence of
keys pressed to create the word.  

\section{Tests}
The following section will test the messages in the \texttt{Messages.hs} file and print the result
in a file. This file is attached in the file. Note that the program will not write the file for the user! The file
just creates output called a, while in the program use \texttt{writeFile "file\_name" a} to create a nicely printed 
file of the results. The file \texttt{test.txt} is attached at the end of this report. 

\end{document}

